\subsection{Выбор преобразователя уровней}

Так как выбранный микроконтроллер использует \textit{3.3V} логику, а субмодуль экрана - \textit{5V}, то для решения проблемы предложено использовать преобразователи уровней.

\subsubsection{Преобразователи уровней}
Преобразователи уровня представляют собой специальные компоненты в цифровых устройствах и предназначаются для согласования сигналов на входе и выходе по току и напряжению, при применении в одном таком устройстве интегральных микросхем из совершенно разных семейств, и уже тем более, когда различны напряжения их питания.

Преобразование уровней можно выполнить несколькими способами:
\begin{itemize}
    \item Аналоговой схемой делителя напряжения
    \item С использованием MOSFET
    \item Используя микросхему
\end{itemize}

Так как каналов, уровень которых необходимо преобразовать достаточно много, будет выбран крайний вариант преобразования уровней. Выбранной логической микросхемой стала \textit{TXS0108EPWR} в корпусе \textit{TSSOP-20}. Эта микросхема преобразовывает 8 каналов, поддерживает необходимые рамки напряжения.