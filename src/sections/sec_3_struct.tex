\section{Разработка структурной схемы}
\label{sec:struct}

\subsection{Постановка задачи}

Для того, чтобы составить структуру разрабатываемого устройства, необходимо выделить функции, которые будет выполнять устройство, затем определить компоненты и связь между ними исходя из данных функций. Результаты можно посмотреть на структурной схеме, представленной в приложении А. 
Задачей данного проекта станет разработка устройства для реализации эмуляции контроллера светодиодного экрана для задач проведения диагностики. Исходя из этого, были выделены следующие функции, которые должно выполнять данное устройство: 
\begin{enumerate}
    \item Обработка входных данных
    \item Формирование сигнала для субмодуля
    \item Переключение режима работы
\end{enumerate}

\subsection{Определение компонентов структуры устройства}

Компоненты структуры устройства выбираются исходя из функций, определенных в постановке задачи. Проанализировав выделенные функции, были определены следующие компоненты, представленные ниже:
\begin{enumerate}
    \item Модуль взаимодействия с внешними устройствами
    \item Микроконтроллер
    \item Модуль питания - стабилизатор и преобразователь напряжения, защита схемы от смены полярности
    \item Модуль индикации
    \item Модуль преобразования уровней
\end{enumerate}

\subsection{Взаимодействие компонентов устройства}

Задача модуля питания - предоставить качественную силовую линию всем потребителям на схеме. К ним относятся микроконтроллер, преобразователи уровней, индикация. Микроконтроллер, в свою очередь, генерирует необходимый сигнал на субмодуль, взаимодействует с модулем внешних устройств (частично интегрирован в МК). Модуль индикации взаимодействует с микроконтроллером и отображает состояние устройства во время его работы.