\sectionCenteredToc{ВВЕДЕНИЕ}
\label{sec:intro}

Семимильными шагами ступает прогресс по планете Земля. Каждый день мы можем наблюдать,
как в мире происходит так называемая «цифровая революция», которая
началась еще в последних десятилетиях прошлого века. Связана она с распространением
информационных технологий и проникновением их во все сферы жизни общества.


В современном мире технологий, где светодиодные экраны стали неотъемлемой частью нашей повседневной жизни, обеспечение надежности и качества работы этих экранов приобретает особую важность. Сценические светодиодные экраны используются в различных сферах, от концертных выступлений и спортивных событий до выставок и мероприятий. Однако, как и любая техническая система, они подвержены износу, поломкам и неисправностям.

Для обеспечения бесперебойной работы сценических светодиодных экранов и своевременной диагностики и устранения неисправностей, было разработано устройство для диагностики таких модулей. Эта система, разрабатываемая с целью оптимизации и упрощения процесса ремонта модулей светодиодных экранов, представляет собой важный шаг в повышении эффективности обслуживания сценической аппаратуры.

В данном проекте будут рассмотрены ключевые аспекты схемотехники и разработки устройства, которые позволят точно эмулировать работу контроллера светодиодных экранов, а также проводить их диагностику без необходимости использования настоящего оборудования. Эта система будет способствовать сокращению времени ремонта, снижению затрат на обслуживание и повышению надежности сценических светодиодных экранов.

Данный курсовой проект предоставит уникальную возможность погрузиться в мир современной схемотехники, а также внедрить инновационные решения в сфере обслуживания светодиодных экранов. Вместе мы создадим систему, способную значительно улучшить качество и доступность сценичной аппаратуры, обеспечивая ее бесперебойную работу на мероприятиях всего мира.