\subsection{Описание алгоритма работы матрицы}
В результате проведенного анализа в предыдущем подразделе, можно обозначить следующий принцип работы:
\begin{enumerate}
    \item выставляем данные на RGB входы, щелкаем клоком CLK. Повторяем, пока не загрузим всю строку;
    \item выключаем выходы OE = 1 (чтобы помех не было);
    \item выдаём на дешифратор номер загруженного ряда;
    \item щелкаем параллельной загрузкой LAT — данные строки переносятся в выходные регистры;
    \item включаем выходы OE = 0;
    \item повторяем для следующего ряда.
\end{enumerate}

После проведения расчетов по количеству необходимой RAM памяти для такой адресации:
\[N_{pixels}=H \cdot V \cdot N_{color} \cdot 8_{bits} = 168 \cdot 42 \cdot 3 \cdot 8 = 169344\text{ бит}\]

Так как микроконтроллер не способен предоставить такой объем оперативной памяти, было принято решение снизить количество бит цвета до 1, а так же применить подход хранения информации в битах используя битовые операции и сдвиги для оптимизации памяти.

\begin{figure}[ht]
    \centering
    \includegraphics[width=0.9\linewidth]{\commonSecPathPrefix/sec_6/content/bit_shifts.png}
    \caption{Расчет хранения информации о цвете пикселей}
\end{figure}
