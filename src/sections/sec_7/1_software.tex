\subsection{Выбор программного обеспечения}
Сюда относятся программы Sprint-Layout, Eagle, DipTrace, ExpressPCB, Altium Designer, TARGET 3001!,
FreePCB, Kicad, DesignSpark PCB, SoloPCB Design, PCB123, TopoR, Pad2Pad, PCB-Investigator, EDWinXP,
Mentor Graphics PADS, ZenitPCB, CADSTAR Express, ZofzPCB 3D Gerber Viewer, PCBWeb, CometCAD, Layo1
PCB, PCB Elegance, NI Ultiboard, CAM350, BoardMaker3, GerberLogix, PCB Artist, VUTRAX, CADintPCB.

\subsubsection{KiCad}
KiCad — это кроссплатформенный комплекс программ с открытым исходным кодом, предназначенный для разработки электрических принципиальных схем и автоматизированной разводки печатных плат. Под обёрткой (логотипом) KiCad содержится изящный пакет следующих программных инструментов:

\begin{itemize}
    \item KiCad : Менеджер проектов.
    \item Eeschema : Редактор электрических схем и компонентов.
    \item CvPcb : Программа выбора посадочных мест для компонентов (всегда запускается из Eeschema).
    \item Pcbnew : Редактор топологии (проводящего рисунка) печатных плат и посадочных мест.
    \item GerbView : Программа просмотра файлов в формате Gerber.

\end{itemize}


Обычно, эти инструменты запускаются из менеджера проектов, но их можно запускать и отдельно.


В настоящее время KiCad считается сложившимся комплексом программ, чтобы использовать его для успешной разработки и сопровождения сложных печатных плат.


В KiCad нет ограничения на размер платы, с его помощью можно разрабатывать платы, содержащие до 32 медных слоёв (слоёв металлизации), до 14 технических слоёв и до 4 вспомогательных слоёв.


Будучи ПО с открытым исходным кодом (лицензируемое GPL), KiCad представляет собой идеальный инструмент для проектов, ориентированных на разработку электронных устройств с открытой документацией.

KiCad доступен для Linux, Windows и Apple OS X (экспериментальная разработка, но работает хорошо).
Файлы и каталоги KiCad
\subsubsection{P-CAD}
P-CAD – мощная САПР, которая состоит из двух автономных модулей – Schematic (редактор электри-
ческих схем) и PCB (редактор печатных плат). Проекты схем могут содержать до 999 листов, а проекты плат –
до 999 слоев размером 60×60 дюймов. Существуют возможности интерактивной разводки дифференциальных
пар для минимизации электромагнитных помех, мультимаршрутная трассировка по заданным параметрам,
ортогональное перетаскивание проводников. Кроме основных подпрограмм P-CAD имеет вспомогательные:
Library Executive (менеджер библиотек), Symbol Editor (редактор символов элементов), Pattern Editor (редактор
посадочных мест, корпусов элементов) и некоторые другие. Библиотеки P-CAD хранят более 27 тысяч
элементов, сертифицированных по стандарту ISO 9001. Полностью поддерживаются форматы Gerber и
ODB++.


Летом 2006 года владелец программы австралийская компания Altium официально заявила, что пре-
кращает развитие P-CAD. Разработчикам было предложено перейти на Altium Designer – более мощный
продукт компанию. Постоянно обновляемые библиотеки Altium Designer хранят более 90 тысяч компонентов.
Многие из них имеют модели посадочных мест, IBIS и SPICE-модели, а также 3D-модели. Каждую из них можно
создать в программе самостоятельно.


Существует возможность разработки печатной платы в трёхмерном виде с импортом/экспортом данных
в механические САПР (SolidEdge, SolidWorks, AutoCAD, ProEngineer). Altium Designer поддерживает
практически все существующие форматы выходных файлов: DXF, Gerber, NC Drill, ODB++, VHDL, IPC-D-356 и
многие другие. Встроенный мастер импорта проектов преобразовывает библиотеки, схемы и платы из систем
OrCAD, P-CAD, Allegro PCB, PADs, DxDesigner в работы Altium Designer независимо от кодировки (бинарной
или ASCII). Отличительной особенностью среды проектирования является сквозная целостность разработки
на разных этапах проектирования. Другими словами изменения, внесённые на любом уровне разработки, будут
отражены на всех стадиях проекта. Демонстрационный проект (см. рисунок 2) показывает три варианта
прошивки для разных ПЛИС. Связь между этапами проекта показана линиями.

Таким образом, рассмотренные ПО схожи друг с другом функционально и отличаются набором
библиотек, интерфейсом и дополнительными опциями. Для того, чтобы определить какое именно ПО
необходимо использовать зависит только от требований, предъявляемых при разработке печатных плат.
Начинающему проектировщику для создания печатных плат подойдут такие пакеты как sPlan, Eagle, DipTrace,
Sprint-Layout, ExpressPCB. Эти программы обладают упрощенным интерфейсом и сравнительно низкой
стоимостью, которая варьируется от $600BYN$ до $4782BYN$. Так же существуют бесплатные версии этих программ с
ограниченными возможностями.


Для более продвинутых пользователей из всего выбора программного обеспечения лучшим вариантом
будут такие программы, как Proteus и AutoCAD Electrical из CAD-программ и Altium Designer как программа для
проектирования печатных плат. Эти программы обладают огромным выбором библиотек, необходимым
функционалом, дополнительными модулями, удобны в использовании. Стоимость этих программ является не
маленькой (лицензия Altium Designer составляет примерно от $6000BYN$), однако из-за достоинств и возможностей
программы приобретение такого пакета является экономически выгодным.